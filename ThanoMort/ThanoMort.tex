%%This is a very basic article template.
%%There is just one section and two subsections.
\documentclass{article}

%%%%%%%%%%%%%%%%%%%%%%%%%%%%%%%%%%%%%%%%%%%%%%%%%%%%%%%%%%%%%%%%%%%%%
% header, footers for review
%\usepackage{fancyhdr}
%\pagestyle{fancy}

%\chead{submitted as a relationship report to be included in the Special
%Collection ``Formal Relationships'' edited by Joshua R. Goldstein and James W.
%Vaupel}
%\lfoot{Submission \#2518 for review}
%\rfoot{}
%\fancyhf{}
%\fancyhead[C]{\footnotesize{submitted as a relationship report to be included
% in the Special Collection ``Formal Relationships'' edited by Joshua R. Goldstein and James W.
%Vaupel}}
%\fancyfoot[R]{\footnotesize{\bfseries{page \thepage}}}
%\fancyfoot[L]{\footnotesize\bfseries{Submission \#2518 for review}}
%\renewcommand{\headrulewidth}{0pt}
%\renewcommand{\footrulewidth}{0pt}
%\setlength\headheight{117.89105pt}
%\setcounter{page}{2}
%\pagestyle{fancy}

%\makeatletter
%\let\ps@plain\ps@fancy 
%\makeatother

%%%%%%%%%%%%%%%%%%%%%%%%%%%%%%%%%%%%%%%%%%%%%%%%%%%%%%%%%%%%%%
\usepackage{amsmath}
\usepackage{caption}
\usepackage{placeins}
\usepackage{graphicx}
\usepackage{natbib}
\usepackage{setspace}
\usepackage{hyperref}
\doublespacing
\bibpunct{(}{)}{,}{a}{}{;} 
\usepackage{url}
% for the d in integrals
\newcommand{\dd}{\; \mathrm{d}}
\newcommand{\ec}{\quad\quad\text{,}}
\newcommand{\ep}{\quad\quad\text{.}}
% end preamble.

\begin{document}
\title{Mathematical relationships in survival by remaining years of life}
\author{Riffe \& Missov, the dynamic duo}
\maketitle

Hi Trifon,
Everything pending here. First the R scripts associated with the figures and
formulas to follow make use of some pacages that are posted on github, but which
can be loaded straight into R using the devtools package. That's easiest if
you're on a Linux system\ldots

I use $^\star$ to mark thanatological functions where reasonable.
\section{$\mu (a) \rightarrow \mu (y)$}

The tidy approximation is:
\begin{equation}
\mu^\star (y) = \int _{a=0}^\infty \mu (a) \mu (a+y) \frac{l(a+y)}{l(a)} \dd
a\ec
\end{equation}
and Figure~\ref{fig:my} shows what it looks like (logged), where 0 means
destined to die within a year and 100 is 100 years from death.

\begin{figure}[ht!]
\caption{$m(y)$ 2010 US males}
\label{fig:my}
\includegraphics[width=5in]{Figures/my.pdf}
\end{figure}

This is a curious pattern for a few reasons. First, it looks different from
everything in demography that I recognize (but then I don't have some entropy
patterns in my head). Second, under thanatological age, you can only die if
you're age 0, and you can't die if you're of a greater age. All you can do is get closer to zero as time passes. So I can't make sense of positive values. Third, in a
stationary population, the force of mortality by years lived is equal to the force of increment by years left (see
attached paper under review --- sorry in case they asked you to review it and
this is awkward). Birth and death in stationary populations are strangely
convoluted, but I can't make sense of this shape in terms of either. Now let's
look at a surface of the same Figure~\ref{fig:MY}, just because (best viewed in
Adobe\ldots):
\begin{figure}[ht!]
\caption{$m(y)$ 2010 US males}
\label{fig:MY}
\includegraphics[width=5in]{Figures/MY.pdf}
\end{figure}
Everything appears freakishly constant except for some major shifts between $30
>= y <= 85$. Other shifts are comparably subtle, but not necessarily
insignificant. I think part of why I can't interpret this curve is that it is
highly dependant on the closeout age of the lifetable. If we extended $\mu (a)$
to age 150, then these values would just pile onto the curve and not add
information. So I guess $\mu (a)$ is not useful unless qualified in some way or
further transformed. 

\section{$d(a) \rightarrow d(y)$}
We already know that $l(a) = l\star(y)$ for $a=y$, a result from Nicolas Brouard
(and later Jim). In this sense, $d(a)$ is at once the distribution of deaths in
a stationary population, the distribution of age at death for a newly born
cohort, the distribution of time since birth of those dying in a stationary
population (a death cohort), and in general the distribution of lifespans for a
given birth or death cohort in a stationary population. Plenty of
interpretations. If we already have a thanatological interpretation of
$d^\star(y)$, which is indeed not even transformed from $d(a)$ what does it mean
to do the following?:
\begin{equation}
\label{eq:dy}
d^{?}(y) = \int _{a=0}^\infty d(a)\mu(a+y)\frac{l(a+y)}{l(a)} \dd a
\end{equation}
The ``?'' is an intentional superscript. Let's just call this
thing $d^{\star\star}(y)$ and pretend we know what we're referring to.
Figure~\ref{fig:dy} shows what it looks like:

\begin{figure}[ht!]
\caption{$d^{\star\star}(y)$ 2010 US males}
\label{fig:dy}
\includegraphics[width=5in]{Figures/dy.pdf}
\end{figure}

Figure~\ref{fig:lndy} is the same thing again, logged. Also, not somethign I
recognize imediately (there's always the risk of these things being identical
to otherwise known measures from some other context, but they're all pretty
fresh to me).

\begin{figure}[ht!]
\caption{$ln(d^{\star\star}(y))$ 2010 US males}
\label{fig:lndy}
\includegraphics[width=5in]{Figures/lndy.pdf}
\end{figure}

\FloatBarrier

\section{So what?}
Well, step 1 is to interpret these uncanny patterns and give meaning to them.
I'm not sure how hard it's worth trying, as they may indeed be nonsensical, but
it's just a question of adequate head-scratching. Any of the above formulas can
be easily re-written in terms of any single lifetable function, in case that
helps. Equation~\eqref{eq:dy} would look like:

\begin{equation}
\label{eq:dymu}
d^{\star\star}(y) = \int _{a=0}^\infty
\frac{d}{da}\left(e^{-\int_{0}^{a}\mu(x)\dd
x}\right)\mu(a+y)\frac{e^{-\int_{0}^{a+y}\mu(x)\dd x}}{e^{-\int_{0}^{a}\mu(x)\dd
x}}\dd a
\end{equation}

Ugly, but maybe simplifiable? 

These are just some first thoughts. All of this work, including the latex file,
R code, etc, can be found here: \url{https://github.com/timriffe/ThanoMort/}
You can either clone it usng \texttt{git} or just download it as a zip file and
unack it, whatever you prefer.

Best wishes, and lots of fun,

Tim
\end{document}
